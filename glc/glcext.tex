\documentclass[10pt, a4paper]{article}
\usepackage{fontspec} 

% DOCUMENT LAYOUT
\usepackage{geometry} 
\geometry{a4paper, textwidth=5.5in, textheight=8.5in, marginparsep=7pt, marginparwidth=.6in}
\setlength\parindent{0in}

% FONTS
\usepackage[usenames,dvipsnames]{color}
\usepackage{xunicode}
\usepackage{xltxtra}
\defaultfontfeatures{Mapping=tex-text}
\setromanfont [Ligatures={Common}, Numbers={OldStyle}, Variant=01]{Linux Libertine O}
\setmonofont[Scale=1.0]{LMMono10}

% ---- CUSTOM COMMANDS
\chardef\&="E050
\newcommand{\html}[1]{\href{#1}{\scriptsize\textsc{[html]}}}
\newcommand{\pdf}[1]{\href{#1}{\scriptsize\textsc{[pdf]}}}
\newcommand{\doi}[1]{\href{#1}{\scriptsize\textsc{[doi]}}}
% ---- MARGIN YEARS
\usepackage{marginnote}
\newcommand{\amper{}}{\chardef\amper="E0BD }
\newcommand{\years}[1]{\marginnote{\scriptsize #1}}
\renewcommand*{\raggedleftmarginnote}{}
\setlength{\marginparsep}{7pt}
\reversemarginpar

% HEADINGS
\usepackage{sectsty} 
\usepackage[normalem]{ulem} 
\sectionfont{\mdseries\upshape\Large}
\subsectionfont{\mdseries\scshape\normalsize} 
\subsubsectionfont{\mdseries\upshape\large} 

% PDF SETUP
% ---- FILL IN HERE THE DOC TITLE AND AUTHOR
\usepackage[dvipdfm, bookmarks, colorlinks, breaklinks, 
% ---- FILL IN HERE THE TITLE AND AUTHOR
	pdftitle={Giovanni Luca Ciampaglia - vita},
	pdfauthor={Giovanni Luca Ciampaglia},
	pdfproducer={http://www.inf.usi.ch/phd/ciampaglia/}
]{hyperref}  
\hypersetup{linkcolor=blue,citecolor=blue,filecolor=black,urlcolor=MidnightBlue} 

% DOCUMENT
\begin{document}
{\LARGE Giovanni Luca Ciampaglia}\\[1cm]
Bertastraße 36 / 8003 Zürich / Switzerland\\[.2cm]
Phone: \texttt{+41 43 817 20 75}\\
Mobile: \texttt{+41 79 718 81 57}\\
Email: \href{mailto:ciampagg@usi.ch}{ciampagg@usi.ch}\\
\textsc{url}: \href{http://www.inf.usi.ch/phd/ciampaglia/}{http://www.inf.usi.ch/phd/ciampaglia/}\\[.2cm]
Born:  July 12, 1980\\
Nationality: Italian\\
Swiss Residence Permit Type: B

\section*{Areas of specialization}
Computer science • Social computing

\section*{Areas of competence}
Agent-based modeling • Large-scale data analysis • Simulation

\section*{Education}
\noindent
\years{2006-2011}\textbf{Ph.D. candidate in Informatics}\\
Università della Svizzera Italiana, Via G. Buffi 13, 6900 Lugano, Switzerland\\
\textsc{Advisors}: Prof. Luca Maria Gambardella (IDSIA), Dr. Alberto Vancheri \& Dr. Paolo Giordano (USI)\\
\textsc{Expected graduation}: November 2011\\[.2cm]
\years{2000-2006}\textbf{Laurea in Computer science} (B.Sc. \& M.Sc.)\\
Sapienza University of Rome, P.le Aldo Moro 5, 00185 Rome, Italy\\
\textsc{Advisor}: Prof. Brunello Tirozzi (Sapienza)\\[.2cm]
\years{2000}\textbf{Maturità Scientifica} (Secondary education degree)\\
Scientific Lyceum ``Isacco Newton'', V.le Manzoni 74, 00185 Rome, Italy\\[.2cm]

\section*{Appointments held}
\subsection*{Research}
\years{2011}\textbf{ETH Zürich} / Research associate\\
Chair of Sociology, in particular Modeling and Simulation,\\
Clausiusstraße 50, 8092 Zürich, Switzerland\\
\textsc{Supervisor}: Prof. Dr. Dirk Helbing.

\subsection*{Teaching}
\years{2010}\textbf{ETH Zürich} / Lecturer\\
\emph{Lecture with Computer Exercises: Modelling and Simulating Social Systems with \textsc{matlab}} (30 h)\\[.1cm]
\years{2008}\textbf{Università della Svizzera Italiana} / Teaching assistant\\
Bachelor program in Informatics\\
\emph{Artificial Intelligence and Knowledge Management} (18 h)\\[.1cm]
\years{2007}\textbf{Università della Svizzera Italiana} / Teaching assistant\\
Bachelor program in Informatics\\
\emph{Computer Networking} (18 h)\\[.1cm]

\section*{Grants \& awards}
\years{2010}\textsc{Visiting studentship} (3 months) ETH Zürich, Switzerland\\
\years{2008}\textsc{Admission to summer school} (1 month) SFI Complex Systems Summer School, Santa Fe, USA.\\
\years{2006}\textsc{Doctoral studentship} (4 years) Swiss National Science Foundation\\

\section*{IT \& programming}
Programming languages: C/C++, Java, MATLAB\\
Scripting languages: Javascript, Python (SciPy + NumPy, Cython), PHP, Bash, ZSH\\
Markup languages: XHTML, CSS, XML\\
Database systems: MySQL, PostgreSQL, SQLite\\
Web Frameworks: Django, jQuery\\
Revision control: Darcs, SVN, Git\\
Digital typography: \LaTeX, \XeTeX\\
Batch systems: PBS\\
Favourite editor: Vim\\
Personal productivity system: GTD\textsuperscript{®}

\section*{Languages}
Italian (native speaker)\\
English (professional proficiency)\\
French (basic proficiency)

\section*{Publications \& talks}

\subsection*{Peer-reviewed conference papers}
\noindent
\years{2010b}Dario Taraborelli, \textbf{Giovanni Luca Ciampaglia} (2010), ``Beyond notability. Collective deliberation on content inclusion in Wikipedia'', \emph{Fourth IEEE intl conf. on self-adaptive and self-organizing systems workshops} September 27-28 2010, Budapest, Hungary. \doi{10.1109/SASOW.2010.26}\\
\years{2010a}\textbf{Giovanni Luca Ciampaglia}, Alberto Vancheri (2010), ``Empirical analysis of user participation in Online communities: the case of Wikipedia'', \emph{Fourth intl AAAI conf. on weblogs and social Media}, May 23-26 2010, Washington D.C., USA. \pdf{http://www.aaai.org/ocs/index.php/ICWSM/ICWSM10/paper/download/1517/1861}\\ 
\years{2008}\textbf{Giovanni Luca Ciampaglia}, Brunello Tirozzi (2008) ``Go east: a residential land use model for the periphery of Rome'', \emph{Intl cong. on environmental modelling and software}, July 7-10, 2008, Barcelona, Spain. \pdf{http://www.inf.usi.ch/phd/ciampaglia/papers/RomeModel2008.pdf}\\
\years{2007}Rachid Rebiha, \textbf{Giovanni Luca Ciampaglia} (2007), ``An ant colony verification algorithm'' \emph{Intl conf. on intelligent systems design and applications}, October 22-24 2007, Rion de Janeiro, Brazil. \pdf{http://www.inf.usi.ch/phd/ciampaglia/papers/RebihaACOVerification07.pdf}\\

\subsection*{Unpublished works \& Working papers}
\noindent
\years{2008}A. Millner, \textbf{G. L. Ciampaglia}, N. Aravamudan, J. Foster, ``The cost of strategic adaptation in a simple conceptual model of climate change.'' Working Paper for the Santa Fe Institute Complex Systems Summer School. \pdf{http://www.santafe.edu/events/workshops/images/3/32/Simpleclimatechange.pdf}\\
\years{2006}\textbf{Giovanni Luca Ciampaglia}, MSc Thesis in Computer Science, ``Modello di sviluppo urbano mediate automi cellulari e sistemi multi-agente''.\\

%\vspace{1cm}
\vfill{}
%\hrulefill

\begin{center}
{\scriptsize  Last updated: \today\- •\- 
% ---- PLEASE LEAVE THIS BACKLINK FOR ATTRIBUTION AS PER CC-LICENSE
Typeset in {\fontspec{Nimbus Roman No9 L} \XeTeX}\\
\href{http://www.inf.usi.ch/phd/ciampaglia}{http://www.inf.usi.ch/phd/ciampaglia}}
\end{center}

\end{document}
