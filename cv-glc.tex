% :vim:let g:Tex_CompileRule_pdf='xelatex -interactionmode=nonstop $*'
\documentclass[10pt, letterpaper]{article}

% refresh version information and import VC macros
\immediate\write18{sh ./vc}
\input{./vc}

\usepackage{fontspec} 

% DOCUMENT LAYOUT
\usepackage{geometry} 
\geometry{a4paper, textwidth=5.5in, textheight=8.5in, marginparsep=7pt, marginparwidth=.6in}
\setlength\parindent{0in}

% FONTS
\usepackage{xunicode}
\usepackage{xltxtra}
\usepackage[usenames,dvipsnames]{xcolor} % fontspec loads color
\defaultfontfeatures{Mapping=tex-text}
\setromanfont [Ligatures={Common}, Numbers={OldStyle}, Variant=01]{Linux Libertine O}
\setmonofont[Scale=1.0]{Latin Modern Mono}

% ---- CUSTOM COMMANDS
\chardef\&="E050
\newcommand{\html}[1]{\href{#1}{\scriptsize\textsc{[html]}}}
\newcommand{\pdf}[1]{\href{#1}{\scriptsize\textsc{[pdf]}}}
\newcommand{\doi}[1]{\href{#1}{\scriptsize\textsc{[doi]}}}

% ---- MARGIN YEARS
\usepackage{marginnote}
\newcommand{\amper{}}{\chardef\amper="E0BD }
\newcommand{\years}[1]{\marginnote{\small #1}}
\renewcommand*{\raggedleftmarginnote}{}
\setlength{\marginparsep}{2em}
\reversemarginpar

% HEADINGS
\usepackage{sectsty} 
\usepackage[normalem]{ulem} 
\sectionfont{\mdseries\upshape\Large}
\subsectionfont{\mdseries\scshape\large} 
\subsubsectionfont{\mdseries\upshape\normalsize} 

% revision info in the page footer
\usepackage{fancyhdr}
\pagestyle{fancy}
\fancyhf{}
\fancyfoot[RF]{\textcolor{gray}{\footnotesize Rev. \VCRevision\\\GITAuthorDate}}
\fancyfoot[CF]{\thepage}
\renewcommand{\headrulewidth}{0pt}
% also, add \thispagestyle after \maketitle to override class prescription 

% BIB
\usepackage[backend=biber,style=authortitle,maxnames=10,sorting=ydnt,dashed=false]{biblatex}

\addbibresource{personal.bib}

% PDF SETUP
\usepackage[bookmarks, colorlinks, breaklinks, 
	pdftitle={Giovanni Luca Ciampaglia - vita},
	pdfauthor={Giovanni Luca Ciampaglia},
	pdfproducer={http://www.inf.usi.ch/phd/ciampaglia/}
]{hyperref}  
\hypersetup{linkcolor=blue,citecolor=blue,filecolor=black,urlcolor=MidnightBlue} 

\nocite{*}

% DOCUMENT
\begin{document}
{\LARGE Giovanni Luca Ciampaglia}\\[1cm]
919 E 10th St \\
Bloomington, IN 47408\\[.2cm]
Phone: \texttt{1-812-855-7261}\\
Email: \href{mailto:gciampag@indiana.edu}{glciampagl@gmail.com}\\
% Github: \href{http://github.com/junkieDolphin}{junkieDolphin} (more code samples available upon request)\\
\textsc{url}: \href{http://www.glciampaglia.com/}{www.glciampaglia.com}\\[.2cm]
Gender: Male\\
Born:  July 12, 1980\\
Nationality: Italian

\section*{Current Affiliation}

Center for Complex Networks and Systems Research\\
School of Informatics and Computing\\
Indiana University Bloomington

\section*{Areas of specialization}
Complex systems • Collective social phenomena • Complex networks

\section*{Competences}
Modeling • Big data • Simulation

\section*{Education}
\noindent

\years{2012}\textbf{Ph.D. in Informatics}\\
%
\textsl{Università della Svizzera Italiana}, Lugano, Switzerland\\
%
\textsc{Supervisors}: Luca Gambardella (IDSIA), Alberto Vancheri (SUPSI), Paolo
Giordano (TU Wien)\\[1em]

\years{2006}\textbf{\textsl{Laurea} in Computer Science (MSc equivalent)}\\ 
%
Sapienza University, Rome, Italy\\
%
\textsc{Supervisor}: Brunello Tirozzi \\[1em]

\years{1999}\textbf{\textsl{Maturità Scientifica}}\\
%
\textsl{Liceo Scientifico Isacco Newton}, Rome, Italy

\section*{Research}

\years{2013-present}\textbf{Indiana University (Bloomington, USA)}\\
%
Postdoctoral fellow\\
%
Center for Complex Networks and Systems Research (Prof. Filippo Menczer)\\

\years{Jul.-Aug. 2011}\textbf{Wikimedia Foundation, Inc. (San Francisco, USA)}\\
%
Research intern\\
%
Wikimedia Summer of Research, Community Department (Dr. Diederik Van Liere)\\

\years{2010-2011}\textbf{Swiss Federal Institute of Technology (Zürich, Switzerland)}\\ 
%
Research associate\\
%
Chair of Sociology, in particular Modeling and Simulation (Prof. Dirk Helbing)

\section*{Teaching}

\years{2015}\textbf{Indiana University Bloomington (USA)}\\
%
Guest lecturer, School of Informatics and Computing\\
%
\emph{I690 Mathematical Modeling of Complex Systems: Introduction to Game Theory}\\

\years{2013}\textbf{Indiana University Bloomington (USA)}\\
%
Guest lecturer, School of Informatics and Computing\\
%
\emph{I690 Mathematical Modeling of Complex Systems: Agent-based models}\\

\years{2010-2011}\textbf{Swiss Federal Institute of Technology  (Zürich,
Switzerland)}\\ 
%
Lecturer, Program in Humanities, Social and Political Sciences\\
%
\emph{Lectures with Computer Exercises: Modelling and Simulating Social Systems
    with \textsc{matlab}}\\ 

\years{2007-2008}\textbf{\textsl{Università della Svizzera Italiana} (Lugano,
Switzerland)}\\
%
Teaching assistant, Bachelor Program in Informatics\\ 
%
\emph{Artificial Intelligence and Knowledge Management} (Prof. Luca Maria
Gambardella)\\ 
%
\emph{Computer Networking} (Prof. Antonio Carzaniga)

\section*{Consulting}

\years{2012-2013}\textbf{Wikimedia Foundation, Inc. (San Francisco, USA)}\\
%
Research Analyst\\
%
Technology Department, (Dr. Dario Taraborelli)

\section*{Publications \& talks}

\begin{refsection}
    \nocite{Ciampaglia2013, Ciampaglia2014, Ciampaglia2014a, Ciampaglia2015a, Ciampaglia2015b, Davis2016, an2017reports, ciampaglia2018fighting, ciampaglia2018research, ciampaglia2018how, shao2018anatomy, Tambuscio2018, shao2018spread, sasahara2020social, saebi2020honem, nematzadeh2019information, copeland2019fashion, Bhadani2022}
    \printbibliography[heading=subbibliography,title=Journal articles]
\end{refsection}

\begin{refsection}
    \nocite{Rebiha2007, Ciampaglia2008, Ciampaglia2010, Taraborelli2010, ciampaglia2011bounded, Fulper2014, Ciampaglia2015, Park2016, Sayyadiharikandeh2016, Shao2016, Shiralkar2017, Nakandala2017, shiralkar2017finding, hui2018hoaxy, yamaya2020political, huq2021characterizing, ciampaglia2021fact, sumpter2021remod}
    \printbibliography[heading=subbibliography,title=Peer-reviewed conference proceedings contributions]
\end{refsection}

\begin{refsection}
    \nocite{Ciampaglia2014b, ciampaglia2018digital}
    \printbibliography[heading=subbibliography,title=Book chapters]
\end{refsection}

\begin{refsection}
    \nocite{Ciampaglia2012b, Ciampaglia2012a, ciampaglia2017can,ciampaglia2018misinformation}
    \printbibliography[heading=subbibliography,title=Online reports]
\end{refsection}

\begin{refsection}
    \nocite{Ciampaglia2006, Ciampaglia2011a} 
    \printbibliography[heading=subbibliography,title=Theses]
\end{refsection}


\subsection*{Talks \& presentations}

\years{2015} ACM Conference on Computer-Supported Cooperative Work and Social
Computing (CSCW'15). March 16th. Oral presentation.\\

\years{2015} Advancing an industry/academic partnership model for Open
Collaboration research. CSCW'15 satellite workshop. Vancouver, March 14. Invited
talk.\\

\years{2015} Leibniz Institute for the Social Sciences. March 5th. Invited
talk.\\

\years{2015} CRASSH Symposium on Conspiracy Theories and Democracy. University of
Cambridge, March 3rd. Invited talk.\\

\years{2014} Ostrom Workshop Fall Colloquium, Indiana University, December 1st.
Oral presentation.\\

\years{2014} Network Science Colloquium, Indiana University, October 13. Oral
presentation.\\

\years{2014} 10th European Conference on Complex Systems (ECCS14), IMT Lucca,
September 22. Oral presentation.\\

\years{2012} 3rd Intl Conference on Social Informatics, Singapore Management
University, October 6-8. Oral presentation.\\
%
\years{2010} 4th Intl AAAI Conference on Weblogs and Social Media, George
Washington University, May 23-26. Poster presentation.\\
%
\years{2009} Intl Workshop on Coping with Crises in Complex Socio-Economic
Systems, ETH Z\"urich, June 8-12. Oral presentation.\\
%
\years{2009} USI-UZH p2p Workshop, \textsl{Università della Svizzera Italiana},
June 17. Oral presentation.

%\vspace{1cm}
\vfill{}
%\hrulefill

\begin{center}
{\scriptsize  Last updated: \today\- •\- 
% ---- PLEASE LEAVE THIS BACKLINK FOR ATTRIBUTION AS PER CC-LICENSE
Typeset in {\fontspec{Nimbus Roman No9 L} \XeTeX}\\
\href{http://www.inf.usi.ch/phd/ciampaglia}{http://www.inf.usi.ch/phd/ciampaglia}}
\end{center}

\section*{Grants \& awards}

\years{2012}\textsc{Prospective Researcher Fellowship} (18 Months) Swiss
National Science Foundation.\\
%
\years{2012}\textsc{Best Interactive Visualization} KANTAR “Information is
Beautiful” Awards.\\
%
\years{2011}\textsc{Travel grant award} (SocInfo 2011) Singapore Management
University.\\
%
\years{2010}\textsc{Visiting studentship} (3 months) ETH Zürich, Switzerland.\\
%
\years{2008}\textsc{Admission to summer school} (1 month) Santa Fe Institute
Complex Systems Summer School.\\
%
\years{2006}\textsc{Doctoral studentship} (4 years) Swiss National Science
Foundation.
 
\section*{Service}

\years{2012-present} \textsc{Reviewer}: Entropy, EPJ Data Science, Adv. Comp.
Sys., JASSS, ACM TOIS, IEEE Internet Computing, PLoS ONE, Wikimedia Research
Newsletter.\\ 
%
\years{2013-2014} \textsc{Guest editor}: EPJ Data Science.\\
%
\years{2013} \textsc{Workshop Co-organizer}: Collective Behaviors and Networks,
satellite workshop at ECCS'13.\\
%
\years{2014} \textsc{Publicity chair}: ACM Web Science 2014.\\
%
\years{2014} \textsc{Workshop Co-organizer}: Computational Social Science,
satellite workshop at ECCS'14.

\section*{IT \& programming}

Programming languages: C, Java, MATLAB.\\
%
Scripting languages: Javascript, Python/Cython, Unix Shell, VIM.\\
%
Parallel Computing: PBS, IPython, OpenMP, MPI.\\
%
Statistical analysis: R, SciPy + NumPy.\\
%
Network analysis: graph-tool, NetworkX, Gephi.\\
%
Database systems: MySQL, SQLite.\\
%
Web Frameworks: Django, jQuery.\\
%
Revision control: Git.\\
%
Digital typography: \LaTeX, \XeTeX.\\
%
Personal productivity system: GTD.

\section*{Languages}

Italian (native speaker)\\
%
English (professional proficiency)\\
%
German (basic proficiency)\\
%
French (basic proficiency)

\end{document}
