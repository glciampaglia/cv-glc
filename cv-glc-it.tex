% :vim:let g:Tex_CompileRule_pdf='xelatex -interactionmode=nonstop $*'
\documentclass[10pt, letterpaper, italian]{article}

% refresh version information and import VC macros
\immediate\write18{sh ./vc}
\input{./vc}

\usepackage{fontspec} 

% DOCUMENT LAYOUT
\usepackage{geometry} 
\geometry{a4paper, textwidth=5.5in, textheight=8.5in, marginparsep=7pt, marginparwidth=.6in}
\setlength\parindent{0in}

% FONTS
\usepackage{xunicode}
\usepackage{xltxtra}
\usepackage[usenames,dvipsnames]{xcolor} % fontspec loads color
\defaultfontfeatures{Mapping=tex-text}
\setromanfont [Ligatures={Common}, Numbers={OldStyle}, Variant=01]{Linux Libertine O}
\setmonofont[Scale=1.0]{Latin Modern Mono}

% ---- CUSTOM COMMANDS
\chardef\&="E050
\newcommand{\html}[1]{\href{#1}{\scriptsize\textsc{[html]}}}
\newcommand{\pdf}[1]{\href{#1}{\scriptsize\textsc{[pdf]}}}
\newcommand{\doi}[1]{\href{#1}{\scriptsize\textsc{[doi]}}}

% ---- MARGIN YEARS
\usepackage{marginnote}
\newcommand{\amper{}}{\chardef\amper="E0BD }
\newcommand{\years}[1]{\marginnote{\small #1}}
\renewcommand*{\raggedleftmarginnote}{}
\setlength{\marginparsep}{2em}
\reversemarginpar

% HEADINGS
\usepackage{sectsty} 
\usepackage[normalem]{ulem} 
\sectionfont{\mdseries\upshape\Large}
\subsectionfont{\mdseries\scshape\large} 
\subsubsectionfont{\mdseries\upshape\normalsize} 

% revision info in the page footer
\usepackage{fancyhdr}
\pagestyle{fancy}
\fancyhf{}
\fancyfoot[RF]{\textcolor{gray}{\footnotesize Rev. \VCRevision\\\GITAuthorDate}}
\fancyfoot[CF]{\thepage}
\renewcommand{\headrulewidth}{0pt}
% also, add \thispagestyle after \maketitle to override class prescription 

% BIB
\usepackage[backend=biber,style=authortitle,maxnames=10,sorting=ydnt,dashed=false]{biblatex}

\addbibresource{personal.bib}

% PDF SETUP
\usepackage[bookmarks, colorlinks, breaklinks, 
	pdftitle={Giovanni Luca Ciampaglia - vita},
	pdfauthor={Giovanni Luca Ciampaglia},
	pdfproducer={http://glciampaglia.com/}
]{hyperref}  
\hypersetup{linkcolor=blue,citecolor=blue,filecolor=black,urlcolor=MidnightBlue} 

\nocite{*}

% DOCUMENT
\begin{document}
{\LARGE Giovanni Luca Ciampaglia}\\[1cm]
% \textcolor{gray}{\emph{(Joh-vah-nee Loo-kah Chahm-pa-yaa)}}\\[1cm]
218 South Maple Street\\
Bloomington, IN 47404\\[.2cm]
Tel: \texttt{1-812-855-7261}\\
Email: \href{mailto:gciampag@indiana.edu}{glciampagl@gmail.com}\\
Github: \href{http://github.com/glciampaglia}{glciampaglia} (estratti di codice sorgente disponibili su richiesta)\\
\textsc{url}: \href{http://www.glciampaglia.com/}{www.glciampaglia.com}\\[.2cm]
Genere: Male\\
Data di nascita:  July 12, 1980\\
Nazionalità: Italiana (Detentore di visto J1 per gli Stati Uniti)

\section*{Affiliazione corrente}

Indiana University Network Science Institute\\
School of Informatics and Computing\\
Indiana University Bloomington

\section*{Aree di specializzazione}
Sistemi complessi • Scienza delle reti • Fenomeni sociali collettivi

%\section*{Competences}
%Modeling • Big data • Simulation

\section*{Istruzione}
\noindent

\years{2012}\textbf{Dottorato in Informatica}\\
%
Università della Svizzera Italiana, Lugano, Svizzera\\
%
\textsc{Relatori}: Luca Gambardella (IDSIA), Alberto Vancheri (SUPSI), Paolo
Giordano (TU Wien)\\[1em]

\years{2006}\textbf{Laurea in Informatica (v.o.)}\\ 
%
Università degli Studi di Roma ``La Sapienza''\\
%
\textsc{Relatore}: Brunello Tirozzi \\[1em]

\years{1999}\textbf{Maturità Scientifica}\\
%
Liceo Scientifico Isacco Newton, Roma

\section*{Ricerca}

\years{2015-present}\textbf{Indiana University} (Bloomington, USA)\\
%
Ricercatore (\textsl{Assistant Research Scientist})\\
%
Indiana University Network Science Institute\\

\years{2013-2015}\textbf{Indiana University (Bloomington, USA)}\\
%
Assegnista di ricerca finanziato dal Fondo Nazionale Svizzero per la Ricerca Scientifica\\
%
Center for Complex Networks and Systems Research (Prof. Filippo Menczer)\\

\years{Lug.-Ago. 2011}\textbf{Wikimedia Foundation, Inc. (San Francisco, USA)}\\
%
Ricercatore tirocinante (\textsl{Research intern})\\
%
Wikimedia Summer of Research, Community Department (Dr. Diederik Van Liere)\\

\years{2010-2011}\textbf{Politecnico Federale di Zurigo, Svizzera}\\ 
%
Ricercatore associato\\
%
Cattedra di \textsl{Computational Social Science} (Prof. Dr. Dirk Helbing)

\section*{Insegnamento}

\years{2015}\textbf{Indiana University Bloomington (USA)}\\
%
Docente per la \textsl{School of Informatics and Computing}\\
%
\emph{I690 Modellizzazione Matematica dei Sistemi Complessi: Introduzione alla Teoria dei Giochi}\\

\years{2013}\textbf{Indiana University Bloomington (USA)}\\
%
Docente per la \textsl{School of Informatics and Computing}\\
%
\emph{I690 Modellizzazione Matematica dei Sistemi Complessi: Introduzione ai Modelli Multi-Agente}\\

\years{2010-2011}\textbf{Politecnico Federale di Zurigo, Svizzera}\\ 
%
Docente, Corso di Scienze Umanistiche, Sociali e Politiche\\
%
\emph{Laboratorio Informatico: Modellizzazione e Simulazione dei Sistem Social con \textsc{matlab}}\\ 

\years{2007-2008}\textbf{\textsl{Università della Svizzera Italiana} (Lugano,
Svizzera)}\\
%
Assistente, Corso di \textsl{Bachelor} in Informatica\\
%
\emph{Intelligenza Artificiale e Gestione della Conoscenza} (Prof. Luca Maria
Gambardella)\\ 
%
\emph{Reti di Computer} (Prof. Antonio Carzaniga)

\section*{Industria}

\years{2012-2013}\textbf{Wikimedia Foundation, Inc. (San Francisco, USA)}\\
%
Analista Ricercatore\\
%
Technology Department, (Dr. Dario Taraborelli)

\section*{Pubblicazioni \& interventi}

\begin{refsection}
\nocite{Davis2016,Nematzadeh2016,Tambuscio2016,Ciampaglia2016,Sayyadiharikandeh2016,Simas2014}
\printbibliography[heading=subbibliography,title=In Preparazione]
\end{refsection}

\begin{refsection}
\nocite{Ciampaglia2012,Ciampaglia2014,Ciampaglia2015a,Ciampaglia2015b}
\printbibliography[heading=subbibliography,title=Riviste Internazionali a Revisione Paritaria]
\end{refsection}

\begin{refsection}
\nocite{Rebiha2007,Taraborelli2010,Ciampaglia2011,Ciampaglia2008,Ciampaglia2010,Ciampaglia2015,Menczer2015,Park2016,Shao2016}
\printbibliography[heading=subbibliography,title=Contributi a Conferenze Internazionali con Revisione Paritaria]
\end{refsection}

\begin{refsection}
\nocite{Ciampaglia2012b, Ciampaglia2012a, Ciampaglia2014c, Ciampaglia2014d, Ciampaglia2006, Ciampaglia2011a, Millner2008} 
\printbibliography[heading=subbibliography,title=Tesi \& Miscellanea]
\end{refsection}


\subsection*{Interventi \& presentazioni}

\years{2016}
\href{http://www.poynter.org/2016/whats-does-the-future-of-automated-fact-checking-look-like/404937/}{Tech
\& Check} Conference. Marzo 31. Relazione su invito.\\
%
\years{2015} \href{http://www.ttivanguard.com/}{TTI/Vanguard} ``Collaboration
and the Future of the Workplace''. Ottobre 30. Relazione su invito.\\
%
\years{2015} Conference on Complex Systems (CCS'15). Settembre 28.
Presentazione orale.\\
%
\years{2015} International Conference on Network Science (NetSci'15). Giugno 4.
Presentazione orale.\\
%
\years{2015} ACM Conference on Computer-Supported Cooperative Work and Social
Computing (CSCW'15). Marzo 16. Presentazione orale.\\
%
\years{2015} Advancing an industry/academic partnership model for Open
Collaboration research. CSCW'15 satellite workshop. Vancouver, Marzo 14.
Relazione su invito.\\
%
\years{2015} Leibniz Institute for the Social Sciences. Marzo 5. Relazione su
invito.\\
%
\years{2015} CRASSH Symposium on Conspiracy Theories and Democracy.
University of Cambridge, Marzo 3. Relazione su invito.\\
%
\years{2014} Ostrom Workshop Fall Colloquium, Indiana University, Dicembre 1.
Presentazione orale.\\
%
\years{2014} Network Science Colloquium, Indiana University, Ottobre 13. 
Presentazione orale.\\
%
\years{2014} 10th European Conference on Complex Systems (ECCS14), IMT Lucca,
Settembre 22. Presentazione orale.\\
%
\years{2012} 3rd Intl Conference on Social Informatics, Singapore Management
University, Ottobre 6-8. Presentazione orale.\\
%
\years{2010} 4th Intl AAAI Conference on Weblogs and Social Media, George
Washington University, Maggio 23-26. Presentazione di poster.\\
%
\years{2009} Intl Workshop on Coping with Crises in Complex Socio-Economic
Systems, ETH Z\"urich, Giugno 8-12. Presentazione orale.\\
%
\years{2009} USI-UZH p2p Workshop, \textsl{Università della Svizzera
Italiana}, Giugno 17. Presentation orale.

\section*{Finanziamenti \& premi}

% \years{2015}\textsc{Co-Principal Investigator} (3 years, \$2M, \emph{submitted})
% NSF 15-574: IIS-III (Medium) ``Mining Knowledge Networks for Computational Fact
% Checking''.\\
%
\years{2012}\textsc{Assegno di ricerca post-dottorale} (18 mesi) Fondo
Nazionale Svizzero per la Ricerca Scientifica.\\
%
\years{2012}\textsc{Migliore visualizzazione interattiva} KANTAR “Information
is Beautiful” Awards.\\
%
\years{2011}\textsc{Finanziamento per spese di viaggio} (SocInfo 2011)
Singapore Management University.\\
%
\years{2010}\textsc{Borsa di studio} (3 mesi) Politecnico Federale di Zurigo, Svizzera.\\
%
\years{2008}\textsc{Ammissione a scuola estiva} (1 mese) Santa Fe Institute
Complex Systems Summer School.\\
%
\years{2006}\textsc{Borsa di studio} (4 years) Fondo Nazionale Svizzero per la Ricerca Scientifica.
 
\section*{Servizio}

\years{2016} \textsc{Co-organizzatore}: workshop satellite ``Open Science for an Open Society'' alla Conferenza sui Sistemi Complessi.\\
%
\years{2016} \textsc{Co-organizzatore}: workshop satellite ``Computational Social Science'' alla Conferenza sui Sistemi Complessi.\\
%
\years{2015} \textsc{Co-organizzatore}: workshop satellite ``Computational Social Science'' alla Conferenza sui Sistemi Complessi.\\
%
\years{2015-presente} \textsc{Membro del consiglio consultivo}: Open Collaboration Data
Factory (Finanziamento 144920, National Science Foundation degli USA).\\
%
\years{2014-presente} \textsc{Membro}: SIGWEB.\\
%
\years{2014-presente} \textsc{Membro}: Complex Systems Society.\\
%
\years{2014} \textsc{Direttore della pubblicità}: Conferenza ``ACM Web Science'' 2014.\\
%
\years{2014} \textsc{Co-organizzatore}: workshop satellite ``Computational Social Science'' alla Conferenza sui Sistem Complessi.\\
%
\years{2013-2014} \textsc{Curatore ospite}: Rivista internazionale a revisione paritaria ed accesso aperto ``EPJ Data Science''.\\
%
\years{2013} \textsc{Co-organizzatore}: workshop satellite ``Collective Behaviors and Networks'', alla Conferenza Europea sui Sistemi Complessi.\\
%
\years{2012-presente} \textsc{Revisore} per le seguenti riviste o conferenze
internazionali: Scientific Reports, Network Science, Frontiers in ICT, PeerJ
Computer Science, CSCW (\emph{distinguished reviewer}), Entropy, EPJ Data
Science, Adv. Comp. Sys., JASSS, ACM TOIS, IEEE Internet Computing, PLoS ONE, Wikimedia
Research Newsletter.\\ 
%
\years{2014-presente} \textsc{Membro del comitato scientifico} dei seguenti workshop o conferenze internazionali:
\href{http://www.snow-workshop.org/}{SNOW} (2014, 2016), CompleNet (2016),
\href{http://cnets.indiana.edu/cool2014/}{COOL}, CRIMENET (2014), DataWiz
(2014), \href{http://dyad.di.unito.it/}{DYAD} (2014), HyperText (2015), SocInfo
(2014, 2016).\\

% \section*{IT \& programming}
% 
% Programming languages: C, Java, MATLAB.\\
% %
% Scripting languages: Javascript, Python/Cython, Unix Shell, VIM.\\
% %
% Parallel Computing: PBS, IPython, OpenMP, MPI, PIG, Hive.\\
% %
% Statistical analysis: R, SciPy + NumPy.\\
% %
% Network analysis: graph-tool, NetworkX, Gephi.\\
% %
% Database systems: MySQL, SQLite.\\
% %
% Web Frameworks: Django, jQuery.\\
% %
% Revision control: Git.\\
% %
% Digital typography: \LaTeX, \XeTeX.\\
% %
% Personal productivity system: GTD.

\section*{Lingue}

Italiano (madrelingua)\\
%
Inglese (livello C2)\\
%
Tedesco (livello A1)\\
%
Livello (livello A1)

%\vspace{1cm}
\vfill{}
%\hrulefill

\begin{center}
{\scriptsize  Last updated: \today\- •\- 
% ---- PLEASE LEAVE THIS BACKLINK FOR ATTRIBUTION AS PER CC-LICENSE
Typeset in {\fontspec{Nimbus Roman No9 L} \XeTeX}\\
\href{http://glciampaglia.com}{http://glciampaglia.com}}
\end{center}

\end{document}
