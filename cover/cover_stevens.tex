\documentclass[11pt, a4paper]{article}
\usepackage{fontspec} 
\usepackage{lipsum}

% DOCUMENT LAYOUT
\pagestyle{empty}
\usepackage{geometry} 
\geometry{
    a4paper, 
    textwidth=5.5in,
    textheight=8.5in,
    marginparsep=7pt, 
    marginparwidth=.6in
}
\setlength{\parskip}{1em}
\setlength{\parindent}{1em}

% FONTS
\usepackage[usenames,dvipsnames]{color}
\usepackage{xunicode}
\usepackage{xltxtra}
\defaultfontfeatures{Mapping=tex-text}
\setromanfont [Ligatures={Common}, Numbers={OldStyle}, Variant=01]{Linux Libertine O}
\setmonofont[Scale=1.0]{LMMono10}

% PDF SETUP
% ---- FILL IN HERE THE DOC TITLE AND AUTHOR
\usepackage[dvipdfm, bookmarks, colorlinks, breaklinks, 
% ---- FILL IN HERE THE TITLE AND AUTHOR
	pdftitle={Giovanni Luca Ciampaglia - cover letter},
	pdfauthor={Giovanni Luca Ciampaglia},
	pdfproducer={http://www.inf.usi.ch/phd/ciampaglia/}
]{hyperref}  
\hypersetup{linkcolor=blue,citecolor=blue,filecolor=black,urlcolor=MidnightBlue} 

\begin{document}

% SENDER ADDRESS 
\center
\textsc{Giovanni Luca Ciampaglia\\
Bertastraße 36\\
\oldstylenums{8003} Zürich, Switzerland\\
+41 79 718 81 57\\
\href{mailto:ciampagg@usi.ch}{ciampagg@usi.ch}
}

\raggedright
\vspace{0.5in}
\today\\[2em]

% RECIPIENT ADDRESS
\textsc{gsadi} -- Analytical Sociology and Institutional Design\\
Department of Sociology\\ 
B Building Campus UAB \\
08193 Bellaterra (Cerdanyola del Vallès)\\
\vspace{2em}

% TEXT -- ADD YOUR LETTER HERE

My research is about web communities, in particular commons-based peer
production systems. During my doctoral studies I have worked on two problems
related to peer production. The first is the problem of norm formation and how
it impacts user participation, and to a more general extent, community
formation. The second is the problem of collective deliberation and task
allocation.

In the first work I developed an agent-based model of norm formation in a peer
production community, inspired by continuous models of opinion dynamics under
bounded confidence, and compared it against empirical data on the activity
lifespan of editors from the communities of five of the largest language
versions of Wikipedia. The calibration of the model was performed using a
surrogate model of the computer code based on Gaussian process regression. This
work forms the core of my doctoral dissertation.

The second work originates from a collaboration with Dr. Dario Taraborelli
(CRESS, University of Surrey). We analyzed a sample of 200,000+ Article for
Deletion (AfD) discussions from the English Wikipedia and explored the existence
of possible biases in the discussion process originating from the structure of
the discussions and from the polarization of editors into different factions.
This work led also to the creation of an
infographic\footnote{\url{http://notabilia.net}}, which visualizes the pattern
of discussions of the 100 longest AfD discussions.

Even though my education is in computer science, my research interests are
aimed at a strong multidisciplinary curriculum; therefore I am interested in
joining your group.

\vfill
\raggedleft
With my best regards,\\
\emph{Giovanni Luca Ciampaglia}

\end{document}
